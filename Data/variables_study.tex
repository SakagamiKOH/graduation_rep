\documentclass{jsarticle}
\usepackage[utf8]{inputenc}

\usepackage[dvipdfmx]{color,graphicx}
\usepackage{ascmac}
\usepackage{amsmath,amssymb}
\usepackage{epsf}
\usepackage{float}
\usepackage{cite}
\usepackage{geometry}
\usepackage[onehalfspacing]{setspace}
\usepackage{lscape}
\usepackage{url}
%\usepackage{natbib}
\usepackage[flushleft]{threeparttable}
\usepackage{booktabs,,tabulary}
\usepackage{comment}
\usepackage{here}
\usepackage{subfig}
\usepackage{indentfirst}
\usepackage{newtxtext}
\usepackage[absolute,overlay]{textpos}


\newcommand{\footremember}[2]{%
        \footnote{#2}
        \newcounter{#1}
        \setcounter{#1}{\value{footnote}}%
}

% Timesのパラメタ値を書き込む
\newcommand*{\myTeX}{\mbox{T\hspace{-0.132em}%
  \raisebox{-0.171em}{E}\hspace{-0.084em}X}}


\begin{document}

\title{変数調べ}
\author{坂上幸\footremember{alley}{横国経済学部相馬ゼミ Email:
\protect \url{sakagamikoh@gmail.com}}}
\date{\today}

\maketitle

\section{HOURWAGE変数(賃金)}

\subsection{CODES}

所得分布は右にすそ野が長い。したがって、所得を考える際は上限値を設けないと特異な個人によって分析結果が歪んでしまう。これを防止するために上限値:topcodeを予め設定した。実際、例えば1985-2002年までは、年間所得$100000(週間$1923.07)を超える個人は、実際の観測値ではなくtopcodeを割り当てている。具体的なtopcodeの基準表に関しては参照URLを確認のこと。

\url{https://cps.ipums.org/cps/hourly_earnings_topcodes.shtml}
 
\subsection{Description}

HOURWAGE変数は回答者が現在の仕事から一時間当たりいくら得ているかを表す変数。数字はインタビューで報告されたものである。従って、このデータを使う際はインフレーションをConsumer Price Indexによって調整する必要がある。\\
  
 

調査方法はOutgoing Rotation/Earner Study。調査対象のグループを3つに分け、各グループには4カ月間の調査協力期間と8カ月の調査非協力期間を与える。そしてこれを繰り返しローテーションしている。\\
  
\url{https://cps.ipums.org/cps/outgoing_rotation_notes.shtml}\\
  
\url{https://cps.ipums.org/cps-action/variables/HOURWAGE#description_section}

\subsection{Comparability}

インフレーションの景況を除けば、HOURWAGE変数は時間をまたいで比較可能である。

\subsection{所感}

動学的労働供給モデルでない限り、実質賃金にする必要はなさそう。

\section{UHRSWORKORG(労働時間)}

\subsection{Codes}

3桁の整数値である。\\
  
998 = Don’t Know(回答不明。未回答)\\
  
999 = NIU (Not In 
Universe:調査対象外。例えば子供)

\url{https://cps.ipums.org/cps-action/faq#ques8}\\
  
 
\url{https://cps.ipums.org/cps-action/variables/UHRSWORKORG#codes_section}

\subsection{Description}

UHRSWORKORG変数は、調査対象者が賃金をもらえるメインの仕事に週当たり何時間充てているかを表す変数。例えば、副業をやっている人の場合は、メインの仕事でないから週当たり労働時間に含まない。また、ボランティアをやっている人は賃金をもらっていないから、週当たり労働時間には含まれない。給与の支払い期間が質問され(例えば日給、週給、月給)、時給を質問され、最後にそのレートで通常週の労働時間を質問される。

\url{https://cps.ipums.org/cps/hours_worked_variables_notes.shtm}\\
  
\begin{itemize}
    \item UHRSWORKLY:去年、全ての仕事で週当たり何時間働いたことになるかを表す変数。例えば、本業と副業で合計年間3650時間なら、週当たり労働時間は70時間。
    
    \item UHRSWORK1変数:メインの仕事で週当たり何時間働いているかを表す変数。UHRSWORKLYと異なり、Time frame is unspecified.(どの週を見て「週当たり」とするかが不明瞭。例えば、繁忙期と閑散期では異なるはず。)  1994年以降、雇用された成人市民を対象に基本的な調査として執り行われている。よってやや好ましくない.UHRSWORKORGとの違いについて。UHRSWORKORGはUHRSWORK1とは異なり質問で指定された賃金率で働いた通常の労働時間を表す変数。
    
    \item UHRSWORK2:サブの仕事全てで週当たり何時間働いているかを表す変数。UHRSWORKLYと異なり、Time frame is unspecified.(どの週を見て「週当たり」とするかが不明瞭。例えば、繁忙期と閑散期では異なるはず。)  
    
    \item UHRSWORKT:本業、副業含めた全ての仕事で週当たり何時間働いているかを示す。UHRSWORKLYと異なり、Time frame is unspecified.(どの週を見て「週当たり」とするかが不明瞭。例えば、繁忙期と閑散期では異なるはず。)  


\end{itemize}

\subsection{Comparablity}

この変数は時間をまたいで比較可能である。

\section{SEX変数(性別)}

\subsection{Codes}
カテゴリ変数でMale, Female, NIUの3タイプ\\
  
\url{https://cps.ipums.org/cps-action/variables/SEX#codes_section}

\subsection{Description}
SEX変数はそれぞれの人の性別を表す。ただし、SEXは社会的性別ではなく、生物学的性別を表す。\\
  
\url{https://cps.ipums.org/cps-action/variables/SEX#comparability_section}

\subsection{所感}
Codesのところにある「Not in Univerce」はここではどういう意味?全体で見るとNIUは極端に少ないから、無視していい。分析結果を歪めない大きさ。

\section{NCHILD変数(子供の数)}
\subsection{Codes}
0 = N.I.U. 子供がいない\\
  
9 = 9 or more persons (Topcode). つまり極端に大きな異常値によって分析が歪められないように、9人を上限とし、それ以上の子供がいる場合は一律に「9人」にする。\\
  
\url{https://cps.ipums.org/cps-action/variables/NCHILD#codes_section}
\subsection{Description}
NCHILD変数は各個人と同居している子供の数をカウントしている。ただし、その子供の年齢や配偶者の有無は問わない。また、実子のみならず継子、養子も子供の数としてカウントしている\\
  
\url{https://cps.ipums.org/cps-action/variables/NCHILD#description_section}
\subsection{Comparability}
この変数の比較可能性は、家族関係に関する利用可能な情報に決定づけられている。NCHILD変数 は RELATE (Relationship to household head)変数、 MOMLOC変数、 MOMLOC2変数、 POPLOC変数、 POPLOC2変数 に含まれる情報に基づいている。(つまりこれらの変数を調整して作った変数がNCHILD変数ということ。)これらの変数の年度による違いについては、MOMLOC変数、 MOMLOC2変数、 POPLOC変数、 POPLOC2変数 を参照のこと。(よってNCHILD変数を使う限り問題はない)
\subsection{所感}
子供の年齢などは結構重要なのに、含まれていないのは問題だと思った。なぜなら、例えば幼児がいるのと、成人して働きながら実家暮らしをしているのとでは、労働供給行動もだいぶ違うと思うからだ。子供の年齢については以下の変数が参考になるはず\\
  
\begin{itemize}
    \item YNGCH変数:最年少の子供の年齢を表す変数\\
    \url{https://cps.ipums.org/cps-action/variables/YNGCH#codes_section}
    
    \item ELDCH変数。最年長の子供の年齢を表す変数。\\
    \url{https://cps.ipums.org/cps-action/variables/ELDCH#codes_section}
    
    \item さらに5歳以下の子供、つまり義務教育前の子供の数、NCHILD5は分析に最適かもしれない\\
    \url{https://cps.ipums.org/cps-action/variables/NCHLT5#codes_section}
\end{itemize}

\section{AGE変数(年齢)}
\subsection{Codes}
Top codesを以下のように定めた。
1962-1987年は99, 1988-2002年は90, 2002-2004を80, 2004- 85.\\
\url{https://cps.ipums.org/cps-action/variables/AGE#codes_section}

\subsection{Decsription}
各人の最終誕生日時点での年齢を表す\\
\url{https://cps.ipums.org/cps-action/variables/AGE#description_section}

\subsection{Comparability}
top code小さくない!? どうやらアメリカの感覚で言ったらこれは妥当らしい。

\section{RACE変数(人種)}
\subsection{Codes}
カテゴリー変数。各カテゴリ変数が何を表すのかについては、下記ページの表を参照のこと。\\
\url{https://cps.ipums.org/cps-action/variables/RACE#codes_section}

\subsection{Description}
2003年からは21種類の人種カテゴリ変数から複数選択可能であり、2013年以降は26種類の人種カテゴリ変数から複数選択可能になっている。

\subsection{Comparability}
2013年以降は複数選択の解となりえるものも一つの選択肢にした。例えば、アジア人かつインディアンと自認している人のために「アジアかつインディアン」のカテゴリ変数を増やした。よって2013年以降扱うのが良さげ?

\section{REGION変数(地域)}

\subsection{Codes}
カテゴリー変数。カテゴリ変数が何を表すかは下記URLの表を参照\\
\url{https://cps.ipums.org/cps-action/variables/REGION#codes_section}

\subsection{Description}
アンケート回答者の住所が所在した地域を表す。ただし州をさらに複数まとめてカテゴリ変数としている。イメージとしては、北海道地方、東北地方、関東地方、中部地方日本海側、中部地方東海地方、近畿地方、中国地方、四国地方、九州地方をカテゴリー変数にしている感じ。相当に大雑把なくくりになっている。\\
\url{https://cps.ipums.org/cps-action/variables/REGION#description_section}

\subsection{Comparability}
全ての年で問題なく比較可能である。IPUMS-CPSの相当最初の方はごくまれにTIPOで”State Unknon”がある。\\
\url{https://cps.ipums.org/cps-action/variables/REGION#comparability_section}

\subsection{所感}
カテゴリーは個人特定を避けるためか相当ザックリしたくくりで作られている。

\section{MARST変数(結婚状態)}

\subsection{Codes}
カテゴリー変数。9つのカテゴリーに分類されている。それぞれのカテゴリが何を表すのかは下記の表を参照。N.I.Uは「回答不明」カテゴリーと思われる\\
\url{https://cps.ipums.org/cps-action/variables/MARST#codes_section}

\subsection{Description}
MARST変数は各調査対象者の結婚状態を表す。配偶者と同居しているか否かについての情報も含まれる。\\
\url{https://cps.ipums.org/cps-action/variables/MARST#description_section}

\subsection{Comparability}
この変数は年をまたいで比較可能である

\begin{itemize}
    \item 全ての年で子供は独身扱いして”never-married”カテゴリに分類されている。\item ”Separetaed”カテゴリーには夫婦仲の不和が原因で法的に別居している既婚者のみが含まれる。その他の理由での別居、例えば就職や兵役等が原因の別居はこのカテゴリーに含まれない。
    \item 内縁関係がある人、すなわち事実婚である人は”marrid”カテゴリーに分類
    \item ”Married, spouse absent”カテゴリーは結婚はしているが、その配偶者が施設に入所していて同じ家庭で生活していなかったり、週の大半を別の居住で過ごしている場合を表す\\
    \url{https://cps.ipums.org/cps-action/variables/MARST#comparability_section}
\end{itemize}

\section{EMPSTA変数(就業状態)}

\subsection{Codes}
カテゴリー変数。大きくは、「労働力(働く意思と能力・資格があり、現在働いているか若しくは求職中)」の中の”Employed”, “Unemployed”と「非労働力(働けないまたは働く意思がなく求職していない)」に分けられるよう。より詳しくは下記URLの表を参照のこと。\\
\url{https://cps.ipums.org/cps-action/variables/EMPSTAT#codes_section}

\subsection{Description}
EMPSTA変数は労働力(実際に現在働いている若しくは仕事を探している)でありかつもし労働力であるならば現在雇用されているかを表す。この変数はさらに労働力でない人の活動(例えば:家事、通学)または状態(例えば:軍隊のメンバー)に関する情報も含んでいる。労働力か否かの2値の変数に関してはLABFORCE変数を参照のこと。\\
  
個人の雇用状況は前週の活動に関する一連の質問に対する回答に基づき判断される。有給または利益を得るために何らかの仕事をしている人、あるいは家族経営や農場で15時間以上無休で働いていると回答したした人は”at work”カテゴリーに分類する。前週は働いていなかったが、病気、休暇、悪天候、労働争議などが原因で一時的に仕事や事業を休んでいるだけの人は”has job, not at work last week”カテゴリーに分類。つまり労働力に含んでカテゴリーされている。\\
  
CPSは民間人の失業率を推定するために設計されたため、この調査の本来の雇用形態変数では”Armed force”は”N.I.U(Not in Unoverse)”(=調査対象外)に分類される。\\
  
“Unemployed”とは給与や利益を得る仕事をしておらず、かつ短期間休業している仕事もなく、かつ前週に主な活動として求職をしていたり、過去4週間の間に仕事を探していたかの質問に「はい」と回答したりした人である。調査では失業者をさらに2つに分類している。”Experienced”カテゴリーとは以前働いたことがある失業者であり、”Inexperienced”カテゴリーとは始めての仕事を求めている失業者(日本で言うところのいわゆる「就活浪人」的な存在)の2つである\\
  
”Employed”にも”Unemployed”にも属さない人を”not in labor force”という余剰カテゴリーに分類する。(「余剰カテゴリ」とは「その他カテゴリ」的な意味。)この分類に分けられる人人々は退職していたり、通学や家事などの活動に従事ていたり、6カ月以上の病気による障がい者だったり、就業の見込みがないと確信している(ディスカレッジ・ワーカー)であることが考えられる\\
\url{https://cps.ipums.org/cps-action/variables/EMPSTAT#description_section}

\subsection{Comparability}
EMPSTATは1994年に行われた調査の再設計の影響を受けて変数が変化している。1962年から1993年までは、面接者は最初の質問で、直前の1週間の「主な活動」について、年齢と性別に応じた質問文(「家事・仕事・学校などをしていましたか」)を投げかけていた。労働統計局は、この最初の質問が、1994年以前の女性のパートタイム労働の過少申告を助長していると結論付けている。つまり、女性パートタイムワーカーはこの質問では労働時間を過少に申告する傾向があった。1994年に行われた調査の質問事項の変更は、一時的な解雇により失業者と分類された労働者の数にも影響を与える可能性がある。1994年以前は、このような人々は、前の週に仕事があったにもかかわらず仕事に就かなかった理由についての自由回答式の質問(「先週、なぜ仕事を休んだのですか?) 1994年以降、この質問は "Last week, were you on layoff from a job?" と言い換えられた。よって1994年以降のデータを使うべき。\\

1994年の調査改訂は、非労働力(not in labor force)に関する情報の種類にも影響を与えた。1962年から1993年まで、”major activity”の自由回答は、非労働力の4つのカテゴリー(”keeping house”, “unable to work”, “retired”, “other(specify)”)に分類されていたが、データとして得られる具体的な内容は年によって異なっている。1994年は、”not in labor force”の情報のみであった。1995年以降、非労働力は、退職者、障害者、その他の3つのカテゴリーに分類されている。(Descriptionの記述を参照)\\

1995年以降”not in labor foce”の中の”unable to work, but want to work”カテゴリーはNIUに含まれている。この点が気に食わない場合はWANTJOB変数を参照のこと\\
\url{https://cps.ipums.org/cps-action/variables/EMPSTAT#comparability_section}

\subsection{所感}
\begin{itemize}
    \item NIU(Not in universe)はこの場合何を表す?先週の行動に関して未回答で、雇用形態状況において何の判断も下せない感じ?NIUは相当数存在するから、これはデータとして削除か
    \item LABFORCE変数は”No, not in the labor force”, “Yes, in the labor force”, “NIU”の3つのカテゴリー変数しかなく、これよりも粗い情報だと思った
\end{itemize}

\section{EDUC変数(学歴)}

\subsection{Codes}
カテゴリー変数。各カテゴリー変数が意味することは下記URLの表を参照のこと。\\
\url{https://cps.ipums.org/cps-action/variables/EDUC#codes_section}

\subsection{Description}
EDUC変数は回答者の学歴を、最高学歴または修了学位で示したものである。例えば、10年生に在籍していたが卒業できなかった回答者は、EDUC変数では9年生を修了したものとして分類されている。\\
  
EDUC変数は、HIGRADE変数とEDUC99変数という2つの変数の組み合わせであり、これら2つの変数はそれぞれ異なる方法で学歴を測定している。HIGRADE変数は、1992年以前の年について入手可能であり、回答者の学校の最高学年または大学の修了年を示す。EDUC99は1992年から利用可能であり、高校卒業者を最高学歴または卒業証書によって分類している。(ただしこの記述については不明)

\url{https://cps.ipums.org/cps-action/variables/EDUC#description_section}

\subsection{Comparability}
時系列での比較可能性を最大化するため、ユーザーはEDUC変数の最初の2桁のみを使用すべきである。すべての年において、この2桁の数字は、高校卒業までの最高学年または修業年限を示す。EDUC変数は3桁だが、最後の位の情報は細かい(例えば小学校何年生かなど)から、前処理で情報を落とすべきということ。1992 年以前は、高等教育を何年受けたかは分かっているが、その年数を修了することで学位が取得できたかどうかを知る方法はない\\
\url{https://cps.ipums.org/cps-action/variables/EDUC#comparability_section}

\subsection{所感}
「比較可能性最大化」というのは例えばどの変数との比較を可能にするためにこの操作を行えと言っているのか良く解らない。具体的にどういうこと?ただ、中級マクロデータサイエンスと同じ処理をすれば、特段問題はなさそう、、、

\section{SCHCOLL変数(学歴)}

\subsection{Codes}
カテゴリー変数。高等教育(義務教育以上)に関する質問。全日制高校、定時制高校、全日制大学、定時制大学、高校・大学に入学していない、NIUで表される。\\
\url{https://cps.ipums.org/cps-action/variables/SCHLCOLL#codes_section}

\subsection{Description}
SCHLCOLL変数は、前週に16歳から24歳(ASEC 2013以降は16歳から54歳)の回答者が高校または大学に在籍していたかどうか、また在籍していた場合、全日制か定時制かを表す。休日や季節休暇中(→建国記念日みたいな単発の休日を意味すると考えられる)の大学生や高校生は(前週、主に高校・大学に通っていたから)「はい」と答え、夏休み中に授業を受けていない人は(長期休暇中の前週は高校、大学に通っていないから、)「いいえ」と答える。インタビュアーは、まず前週に学校に在籍していたかどうかを尋ね、次に高校・大学に通っているかを訪ね、最後に定時制か全日制かを尋ねた。IPUMS-CPSでは、これらの回答は1つの変数SCHLCOLLにまとめられている。\\
\url{https://cps.ipums.org/cps-action/variables/SCHLCOLL#description_section}

\subsection{Comparability}
この変数は時間をまたいで比較可能。しかし、2013年以降に調査対処(=universe)が変化している点に注意しなければならない。(実際、2013年以前は調査対象が16~24歳だったけど、2013年以降は16~54歳になっている。)\\
  
あといろいろと比較可能性について言及があるが古いデータ1980年以前を使わないなら問題ない。\\
\url{https://cps.ipums.org/cps-action/variables/SCHLCOLL#comparability_section}

\subsection{所感}
アンケートの前週、高校、大学に通っていたかを聞いたら、夏休み中の学生とか大学に在籍していても、現在高校・大学にいないことにならない?少しナンセンスな質問な気もするが、自分の変数理解の誤解か?ただ、中級マクロデータサイエンスと同じ処理をすれば、特段問題はなさそう、、、

\section{EDCYC変数(在学年数)}

\subsection{Codes}
カテゴリ変数。大学以上の教育・研究機関に何年在籍しているかを表している。より詳しい、カテゴリ変数が何を表すかの説明に関しては下記のURLを参照のこと。\\
\url{https://cps.ipums.org/cps-action/variables/EDCYC#codes_section}

\subsection{Description}
EDCYC変数は少なくともある程度の大学教育を受けたが学士号未満の者が、大学で取得した単位数の年数を示すもの。もっと平たく言うと、例えば僕はまだ学士を取得していないが4年生なのでカテゴリー5の”Four or more years”カテゴリーに分類される。大学学部生に関するより詳細な情報を与える変数と考えられる。\\
\url{https://cps.ipums.org/cps-action/variables/EDCYC#description_section}

\subsection{Comparability}
この変数は全ての年で比較可能である。特に変数の再設定等は行われていない模様。\\
\url{https://cps.ipums.org/cps-action/variables/EDCYC#comparability_section}

\section{JTYEARS(勤続年数)}

\subsection{Code}
この変数には小数点以下2桁にアンケート結果集計者が作った決まりがある。IPUMSコマンドファイルは自動的に必要な調整を行うから、それ以上の調整を個別に行う必要はない。\\
  
「アンケート結果集計者が作った決まり」に関しては、\\
99.96 = 回答拒否\\
99.97 = 回答不明\\
99.98 = 未回答\\
99.99 = 空欄のまま\\
だが、いずれの場合も99は欠損値ということだろう。\\
\url{https://cps.ipums.org/cps-action/variables/JTYEARS#codes_section}

\subsection{Description}
要は勤続年数を表す変数。JTYEARS変数は国勢調査局を再コード化したもので、回答者が現在の仕事に従事した年数を示すものである。\\
  
より詳しい説明に関しては2つめのリンクを参照のこと。以下軽く内容を重要なところのみまとめる。\\
  
Job Tenure Supplement(Employee Tenure and Occupational Mobility Supplement : 雇用期間と職業流動性に関する補完調査)はCPSを補完する調査として追加の雇用情報を回答者から集めている。補完調査は回答者の仕事の種類、業界、一年前の仕事、勤続年数などの情報をもたらしている。\\
  
この補足調査は、通常1月か2月に収集され、最も古い形式の調査は、Job Tenure and Occupational Mobility Supplementとして1973年に実施されたものである。IPUMS CPS Job Tenure Supplementのデータは、1983年、1987年、1996年以降のすべての年について入手可能である。\\
  
Job Tenure Supplementの資格要件(→調査対象になる要件)は年によって変化していった。1983年、この調査はJob Tenure/Occupation Mobility and Training Supplementとして実施された。対象者は、基準週に就業していた14歳以上の一般市民、あるいは基準週に失業していたが就業経験がある一般市民。国勢調査局の資料では、失業中の回答者は全体として補足の対象であったが、在職期間変数については対象外である(NIU)ことが示されている。(つまり、在職期間に関する質問は失業者は応えられないからN.I.Uとして記録されているということ。)この資格要件は、本年度のJTRESP変数に反映されている。\\
  
1987年の調査について。14歳以上のすべての一般市民がこの補足調査の対象となった。しかし14歳以上の被雇用者のみがJob Tenure変数の対象であった。(つまり、○○歳で失業中の回答者は勤続年数変数がNIUになっている。)\\
  
1996年以降の補足調査では、基準週に雇用されていた15歳以上の一般市民を調査対象としている。\\
\url{https://cps.ipums.org/cps-action/variables/JTYEARS#description_section}\\
\url{https://cps.ipums.org/cps/jt_sample_notes.shtml}

\subsection{Comparability}
1987年以前も以後も、勤務月数は一貫して収集されていない。1983年と1987年には、現在の仕事で1年未満しか働いていないと答えた回答者だけに、その勤続月数を尋ねる質問をしている。1987年以降では、現在の仕事の勤続年数が3年未満であると答えた回答者に月数を尋ねている。\\
  
Descriptionでも書いたように1996年以降は15歳以上の被雇用者、それ以前は14歳以上の被雇用者を調査対象にしている。\\
  
この変数は、1996年から2012年までのすべての年について完全に比較可能である。2012年からは39のトップコードが導入され(つまり、勤続年数39年以上は外れ値として分析結果を歪めないように39として処理されたということ)、39年以上はすべて39とコード化された。2014年と2016年には、文書には39のトップコードが記載されているが、データには32が真のトップコードである可能性が示されている。(つまり2014,2016年の調査では勤続年数の最大値が32にあるということ。)
\\
\url{https://cps.ipums.org/cps-action/variables/JTYEARS#comparability_section}

\subsection{所感}
Comparabilityの最後の一文の意味がいまいちハッキリしない。”True top code”って何?実際は勤続年数が32年以上の回答者がいたけど、この年はtop codeを32に設定したということ?それとも、これらの年の回答者はみんな勤続年数が32であるから、32が事実上のtop codeになっているってこと?多分、前者が”True top code”の意味だろうけど、、、\\
  
→top codeは本来39である。しかし2014,2016年に限れば、打ち間違いにより(?)、top-codeが32になってしまっている模様。

\section{OCC変数(職業)}

\subsection{Codes}
OCC変数は4桁の数値変数である。(→カテゴリー変数として考えてしまって問題ないだろう。)ただ、それぞれの数値が表す職業については度々改定されているから注意。例えば、一番最新の職業コードについては以下2番目のURLを参照のこと。\\
\url{https://cps.ipums.org/cps-action/variables/OCC#codes_section}\\
\url{https://cps.ipums.org/cps/codes/occ_2020_codes.shtml}\\

\subsection{Description}
OCC変数は、その人の主な職業を報告する。複数の仕事を持つ回答者は、最も多くの時間働いた仕事を報告することになっている。(だから、例え仕事1の方で大きく稼いでいたとしても、仕事2での労働時間が長ければ、回答者は仕事2の方を職業として回答するということ。)調査時に雇用されていた人については、OCC変数は直前の週に働いていた仕事に関連して質問し(だから、先週労働時間が一番長かったものを職業として回答するということ)、失業者や現在非労働力(=not in Labor Force)の人は直近の職業を答えることになっている。CPSの面接者は、その人がどのような仕事をしていたかを尋ねることで情報を収集し、国勢調査局のスタッフがその情報を現代のCPSまたは国勢調査の職業分類にコード化した。1968年以降の一貫した職業分類を使用したいユーザーは、OCC1950変数を使用する必要がある。非労働力の定義を含む雇用概念に関する一般的な議論については、EMPSTATのドキュメントを参照のこと。(→これらの雇用概念については、9番目の変数:EMPSTAで詳しく調べた。)\\
\url{https://cps.ipums.org/cps-action/variables/OCC#description_section}

\subsection{Comparability}
調査対象の変更に加え、CPSの職業データのコード体系も時代とともに変化している。以下で具体的にどれだけ頻繁に改定されてきたかを記す。(1962年から1967年までは、40未満のコードからなる職業分類が使用された。1968-1971年は1960年国勢調査の分類、1972-1982年は1970年国勢調査の分類、1983-1991年は1980年国勢調査の分類、1992-2002年は1990年国勢調査の分類、2003-2010年は2000年国勢調査の分類、2011-2019年は2010年国勢調査の分類、そして2020年から2017年は国勢調査の分類を用いて職業が分類された。このように職業コードは幾度となく改定されてきた。なお、1995年9月以降、1992年から2002年の間に、従来”legislators”(003),”postmasters”(016)とされていたものが”manegers, and administers, N.E.C”(022)と変更され、”judges”(178)とされていたものが”lawyers”(179)と変更されたことに注意が必要である。\\
  
OCC変数は、オリジナルのCPSデータで与えられたコードに基づいている(→どういうこと?)。一方でOCC1950変数は、1950年の国勢調査の分類法を用いて職業を共通のフォーマットに再コード化し、1968年からの経年比較を可能にしている。\\
\url{https://cps.ipums.org/cps-action/variables/OCC#comparability_section}

\subsection{所感}
\begin{itemize}
    \item 現在の非労働力や失業者に対しては前職を聞くということは注意しなければいけないと思った。
    \item ところで、現在非労働力であったり失業中であったりする人で前職がない人(例えば、卒業後ずっと主婦みたいな人?)は”NIU”に分類されるの?
    \item OCC変数はどのようなデータになっているのかよく解らない。基準を最新の2020年基準にしてしまうと、例えば1960年の情報からより細かい職業を復元できないから支障をきたす。しかし、1950年の分類を基準に集計した(より細かい分類は情報を落とすことができるからこれは可能だろう)データはOCC1950変数である。これを踏まえると、OCC変数は、職業分類の改定前後で比較不可能なデータなのか?
\end{itemize}

\section{IND変数(業種変数)}

\subsection{Codes}
IND変数は4桁の数値変数である。

\subsection{Description}
IND変数(=industry)はOCC変数(=occupation)及び、1968年移行はOCC1950変数に記録されているその人が主たる職業を行った産業の種類を報告する。例えば、ある個人iが医者として最も長い時間労働したとする。この場合、OCC変数で”doctor”となる。このOCC変数、もしくはOCC1950変数をもとに個人iの働く産業界を表す変数、OCC変数は「health industry(=医療業界)」として記録されるということ。以上でたとえ話終わり。”Industry”は労働環境と経済部門を意味し、”occupation”は労働者の具体的な技術的機能に関連する。\\
  
調査時に雇用されていた人については、IND変数は回答者が前週に働いていた産業部門を表す。失業者や非労働力の場合は、直近の職場の産業部門を示す。つまり、失業などする前にはどんな業界で働いていたのかを訪ねている。CPS の面接官は、回答者がどのような仕事をしていたかを尋ねることで情報を収集し、国勢調査局のスタッフがその情報を CPS または国勢調査の産業分類にコード化する。1968年以降の一貫した産業コーディングスキームで作業を行いたい人は、IND1950変数を使用する必要がある。\\
\url{https://cps.ipums.org/cps-action/variables/IND#description_section}

\subsection{Comparability}
調査対象の変更に加え、CPSの産業別コード体系も時代とともに変化している。1962年から1967年までは、50未満のコードからなる産業分類スキームが使用された。1968-1970年は1960年国勢調査の分類、1971-1982年は1970年国勢調査の分類、1983-1991年は1980年国勢調査の分類、1992-2002年は1990年国勢調査の分類、2003-2008年は2002年国勢調査の分類、2009-2013年は2007年国勢調査の分類、2014-2019年は2012年国勢調査の分類、2020年から2017年はそれ以前の分類で分類されています。IND変数は、オリジナルのCPSデータで与えられたコードを提供し、IND1950は、1950年の国勢調査の分類スキームを使用して産業を共通のフォーマットに再コード化し、1968年からの経年比較可能性を提供している。

\subsection{所感}
IND変数はどのようなデータになっているのかよく解らない。基準を最新の2020年基準にしてしまうと、例えば1960年の情報からより細かい職業を復元できないから支障をきたす。しかし、1950年の分類を基準に集計した(より細かい分類は情報を落とすことができるからこれは可能だろう)データはOCC1950変数である。これを踏まえると、IND変数は、職業分類の改定前後で比較不可能なデータなのか?IまたND1960,IND1990などあるが、どれを使うべきか?

\section{UNION変数(労働組合)}

\subsection{Codes}
カテゴリ変数。\\
0:調査対象外、\\
1:労組によって保護されていない、\\
2:労組の組員、\\
3:労組の組員ではないが、労組に保護されている\\
\url{https://cps.ipums.org/cps-action/variables/UNION#codes_section}

\subsection{Description}
UNION変数は、回答者が現在の仕事において、1)労働組合または労働組合に類似した従業員団体のメンバーであったか、2)組合員ではないが、労働組合または従業員団体の契約の対象となっていたか、3)組合員でもなく、労働組合契約の対象でもなかったかを示しています。ユーザーは、この変数にEARNWT変数でウェイトを調整する必要がある\\
\url{https://cps.ipums.org/cps-action/variables/UNION#description_section}

\subsection{Comparability}
この変数は、アクセス可能なほぼすべてのサンプルで比較可能である。ただし1986年11月の基本月次サンプルでは、回答者が組合契約の対象であるが組合員でないかどうかを判断するためのデータがなかった。元の公共利用データには、組合員に関するデータのみが存在したからだ

\section{非労働所得}

全て年次データ(各月のデータではない)が、以下のものが参考になると考えられる。
また、年ごとに基準などがそろわず処置が大変そう。\\
\url{https://cps.ipums.org/cps/topcodes_tables.shtml}\\
  
さらに質問フォーマットは以下を参照。このフォーマットで回答者から回答を回収した。\\
\url{https://cps.ipums.org/cps-action/variables/INCUNEMP#questionnaire_text_section}

\subsection{INCSS変数(社会保障)}

\subsubsection{Codes}
数値変数

\subsubsection{Description}
INCSS変数は、回答者が社会保障や米国政府鉄道退職者保険(1968-1979年)、あるいは社会保障のみ(1980年以降)から受け取った税引き前所得(ある場合)を示しています。(例えば、政府から社会保障を月々1万ドルもらったら、INCSS変数は10000となる)\\
  
金額は、面接者に報告された通りに表現されている。ユーザーは、消費者物価指数の調整係数を使用して、インフレを調整する必要がある。つまり名目からインフレ率を考慮した実質に変換する必要がある\\
\url{https://cps.ipums.org/cps-action/variables/INCSS#description_section}

\subsection{INCDIVID変数(株式)}

\subsubsection{Codes}
トップコードなどは年によってまちまち。詳しくは下記参照\\
\url{https://cps.ipums.org/cps/topcodes_tables.shtml}

\subsubsection{Description}
INCDIVID変数はもし存在するならば、前年に税引き前に株式と投資信託によっていくら得たかを表す変数。\\
  
金額は名目値であるから、場合によっては消費者物価指数によって調整して実質値に変換する必要がある。\\
\url{https://cps.ipums.org/cps-action/variables/INCDIVID#description_section}

\subsubsection{Comparability}
インフレの影響を除けば、INCDIVID変数は1988年以降、一貫して比較可能である。つまり1988年以降は名目値として比較可能性が補償されているということ。\\
\url{https://cps.ipums.org/cps-action/variables/INCDIVID#comparability_section}

\subsection{INCRENT変数(家賃・印税など)}

\subsubsection{Codes}
トップコードなどは年によってまちまち。詳しくは下記参照\\
\url{https://cps.ipums.org/cps/topcodes_tables.shtml}

\subsubsection{Description}
INCRENT変数はもし存在するならば、回答者が前年に家賃、下宿料、遺産、信託、印税から得た税引き前所得を表す変数。\\
  
金額は名目値であるから、場合によっては消費者物価指数によって調整して実質値に変換する必要がある。\\
\url{https://cps.ipums.org/cps-action/variables/INCRENT#description_section}

\subsubsection{Comparability}
インフレの影響を除けば、INCDIVID変数は1988年以降、一貫して比較可能である。つまり1988年以降は名目値として比較可能性が補償されているということ。\\
\url{https://cps.ipums.org/cps-action/variables/INCRENT#comparability_section}

\subsection{INCSSI変数(障害・高齢手当)}

\subsubsection{Codes}
トップコードなどは年によってまちまち。詳しくは下記参照\\
\url{https://cps.ipums.org/cps/topcodes_tables.shtml}

\subsubsection{Description}
INCSSI変数は回答者がもしあれば前年にSSI(=Supplement Security Income)から受け取った税引き前所得を表す変数。
(SSIとは障がいがあるこどもや障がいがある大人や65歳以上の老人に対して現金を支給する制度である。)\\
  
金額は名目値であるから、場合によっては消費者物価指数によって調整して実質値に変換する必要がある。\\
\url{https://cps.ipums.org/cps-action/variables/INCSSI#description_section}

\subsubsection{Comparability}
この変数はインフレの影響を除けば、時間をまたいで比較可能である。つまりこの変数は名目値として全て比較できる\\
\url{https://cps.ipums.org/cps-action/variables/INCSSI#comparability_section}

\subsection{INCINT変数(貯金利息)}

\subsubsection{Codes}
トップコードなどは年によってまちまち。詳しくは下記参照\\
\url{https://cps.ipums.org/cps/topcodes_tables.shtml}

\subsubsection{Description}
INCINT変数は回答者がもしあれば、普通貯金、譲渡性預金、債券、国庫債券、IRA、その他の利息から得た税引き前の金額を表す変数である。\\
  
金額は名目値であるから、場合によっては消費者物価指数によって調整して実質値に変換する必要がある。\\
\url{https://cps.ipums.org/cps-action/variables/INCINT#description_section}

\subsubsection{Comparability}
INCINT変数はインフレや調査対象の変化もある上に、質問文も変化している。CPSのインタビューフォームで言及される具体的な利子収入の源泉の種類は、時代とともに増加している。初期には回答者は "貯蓄や債券などの利子 "について質問されていた。1980年以降は、「譲渡性預金、マネー・マーケット・ファンド、債券、国庫債券」、「利子を支払うその他の貯蓄や投資」など、いくつか利子収入の種類が増えた。1988年からは、IRAの記載がさらに追加された。このように、利子所得の源泉をより広範囲に記載するようになったことで、後年、より完全な報告書が作成されるようになったと思われる\\
\url{https://cps.ipums.org/cps-action/variables/INCINT#comparability_section}

\subsection{INCUNEMP変数(ストライキ給付)}

\subsubsection{Codes}
トップコードなどは年によってまちまち。詳しくは下記参照。\\
\url{https://cps.ipums.org/cps/topcodes_tables.shtml}

\subsubsection{Description}
NCUNEMP変数は回答者が前年に州や連邦政府の失業保険、補足失業給付、組合の失業給付またはストライキ給付から受け取った税引き前所得の額をもしあれば表す変数。\\
  
金額は名目値であるから、場合によっては消費者物価指数によって調整して実質値に変換する必要がある。\\
\url{https://cps.ipums.org/cps-action/variables/INCUNEMP#description_section}

\subsubsection{Comparability}
インフレの影響を除けば、この変数は1988年以降、ほぼ一貫していて、時間をまたいで比較可能である。1988年から1994年までは、回答者はその年の数字を1つだけ報告していた。1995年からは、週次、隔週、月2回、月次など、より短い時間間隔の金額と支払い回数を報告することができるようになった\\
\url{https://cps.ipums.org/cps-action/variables/INCUNEMP#comparability_section}

\section{健康状態}

\subsection{HEALTH変数(健康状態)}

\subsubsection{Codes}
カテゴリー変数。5段階で自分の健康状態を申請した\\
\url{https://cps.ipums.org/cps-action/variables/health#codes_section}

\subsubsection{Description}
HEALTH変数は、回答者が現在の健康状態を5段階(Excellent、Very Good、Good、Fair、Poor)で評価した変数である\\
\url{https://cps.ipums.org/cps-action/variables/health#description_section}

\subsubsection{Comparability}
全ての年で完全に比較可能である。\\
\url{https://cps.ipums.org/cps-action/variables/health#comparability_section}

\subsubsection{補足}
各年のアンケート質問文は以下を参照\\
\url{https://cps.ipums.org/cps-action/variables/health#questionnaire_text_section}

\subsection{DIFFCARE変数(長期の病気)}

\subsubsection{Codes}
3つのカテゴリーからなる変数

\subsubsection{Description}
DIFFCARE変数は、回答者が6ヶ月以上続いている身体的または精神的な健康状態にあり、入浴や着替え、家の中での移動など、自分自身の身の回りのことをするのが困難な状況にあるかどうかを表している。骨折や妊娠などの一時的な健康状態は含まれていない\\
\url{https://cps.ipums.org/cps-action/variables/DIFFCARE#description_section}

\subsubsection{Comparability}
この変数は時間をまたいで完全に比較可能である


\section{EARNWT変数}

\subsection{Codes}
EARNWT変数は、4つの暗黙の小数を持つ8桁の数値変数である。例えば、12345678は1234.5678と解釈されることになる。IPUMSのコマンドファイルはEARNWTを自動的に10000で割っているので、それ以上の調整は必要はない。

\subsection{Description}
EARNWT変数は以下の変数のいずれかを含む全ての分析で使用されるべき、個人レベルの重みづけ変数。すなわち、EARNWEEK変数(=回答者が現在の仕事で通常、週当たりいくら稼ぐのかを表す変数)、HOURWAGE変数(=回答者が現在の仕事から一時間あたりいくら稼いでいるかを表す変数)、PAIDHOUR変数(=回答者が現在の仕事で時間給を貰っているかどうかを表すバイナリー変数)、UNION変数、UHRSWORKORG変数(=時間給をもらっている回答者が、通常、主な仕事で1週間に働く総時間数を表す変数)、WKSWORKORG変数(=回答者が通常1年間に働く週数を単週数を表す変数)、ELIGORG変数(=回答者がEarner Studyの対象であるかどうかを表すバイナリー変数)、OTPAY変数(=回答者が残業代、チップ、仲介手数料のいずれかを受け取っているかを示すバイナリー変数)である。\\
  
(EARNWEEK、HOURWAGE、PAIDHOUR、UNION、UHRSWORKORG、WKSWORKORG、ELIGORG、OTPAYを含む)"earner study "の質問をしなかった6グループの個人は、EARNWTの値が0であった。また、"earner study "の質問が行われた2つのローテーショングループでも、軍隊のメンバーはEARNWTの値がゼロである

\end{document}
